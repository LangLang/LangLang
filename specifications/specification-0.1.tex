\documentclass[a4paper,11pt]{article}

%%%%%%%%%%%%%%%%%%% PACKAGES %%%%%%%%%%%%%%%%%%%
%\usepackage{tikz}
%\usepackage{geometry}
%\geometry{a4paper}
%\usepackage{gcl}
%\usepackage{float}
%\usepackage{wrapfig}
\usepackage{mathtools}
%\usepackage{amsmath}
\usepackage{amssymb}
%\usepackage{amsthm}
\usepackage{proof}


%%%%%%%%%%%%%%%%%%%% MACROS %%%%%%%%%%%%%%%%%%%%

%%%%%%%%%%%%%%%%%%% DOCUMENT %%%%%%%%%%%%%%%%%%%
\begin{document}

\title{LangLang 0.1}
%\subtitle{Specification}
\author{Rehno Lindeque}

\maketitle

\begin{abstract}
A formal specification of the syntax and semantics for the programming language \textsl{LangLang}.
\end{abstract}
\textbf{Keywords:} \textsl{LangLang}, specification, denotational semantics, operational semantics

\section{Introduction}

\section{Syntax Specification}
\subsection{Formal syntax}

\subsection{Conventions used in this document}
\label{sec:conventions}

Normally, in source code, \textsl{LangLang}'s syntax would dictate that \emph{bottom} should be written as an empty pair of parentheses \texttt{()} to indicate that \emph{bottom} is formally just an empty collection in \textsl{LangLang}.
However, in this document \emph{bottom} will usually be written using the standard mathematical notation $\bot$.
Similarly \emph{top}, written using an underscore \texttt{\_} in \textsl{LangLang} source code, corresponds to the mathematical notation $\top$ used in this document.\\

In addition the following conventions are used to indicate the nature of syntactic elements in the semantics.

\begin{description}
  \item[$exs$] A collection of expressions (which may also be, depending on circumstance, $\top$, $\bot$ or a single expression).
  \item[$ex$] A collection consisting of a single expression.
  \item[$(e\ es)$] A collection consisting of the single expression $e$ grouped together with the collection of expression $es$.
  \item[$(e_0\ e_1 \dots e_n)$] Another notation for a collection of expressions $e_0$ through $e_n$ where each $e_k$ is a single expression.
  \item[\texttt{t}] A token representing some symbol (atomic unit).
\end{description}

\section{Denotational Semantics}

The denotational semantics listed here serve as invariants that inform the design of the operational semantics in the next section.\\

Atoms are the most basic elements in \textsl{LangLang} and are represented by simple whitespace free \emph{tokens} in source code.
In denotational terms two atoms are equal if their token representation is identical.
To facilitate generic programming tokens may also be constructed programmatically, however we delay this topic for later sections.\\

Collections in \textsl{LangLang} do not have any \emph{structure} as all structure is derived using the concept of arrows.
Intuitively this means that collections are fully defined by their contents and have transient ``boundaries'', so to speak.
They cannot be nested in the sense that any collection that contains another collection is indistinguishable from the union of all their elements.
This is unlike the common notion of \emph{sets} in mathematics which includes the set as a basic mathematical object with its own structure.\\

Syntactically, parentheses are used to group things together in \textsl{LangLang}, thus collections may be represented in code by enclosing their contents.
However, as mentioned above, collections' boundaries are transient and disappear when nested.
\[
(ex_0\ (ex_1\ (ex_2\ ex_3))\ ex_4) \quad\triangleq\quad (ex_0\ ex_1\ ex_2\ ex_3\ ex_4)
\]

Furthermore, when a collection contains only a single expression, parentheses may be dropped entirely.
\[
(ex) \quad\triangleq\quad ex
\]

An empty collection $()$ is denoted, in this document, with the symbol $\bot$\footnote{Please take note of the conventions for writing \emph{bottom} mentioned in section \ref{sec:conventions}.}.
\begin{eqnarray*}
() \quad&\triangleq&\quad \bot \\
(\bot\ \bot) \quad&\triangleq&\quad \bot \\
(ex_0\ ex_1\ \bot\ ex_3) \quad&\triangleq&\quad (ex_0\ ex_1\ ex_3)
\end{eqnarray*}

An arrow is an \emph{operational} construct which adds structure to the only other objects in the language, atoms.
As collections are not objects, arrows may not connect to them directly, but instead connect only to their contents.
For this reason, an arrow that connects to an empty collection simply does not exist.
\begin{eqnarray*}
\bot \rightarrow exs \quad&\triangleq&\quad \bot \\
exs \rightarrow \bot \quad&\triangleq&\quad exs \\
(ex_0\ ex_1) \rightarrow ex_2 \quad&\triangleq&\quad (ex_0 \rightarrow ex_2 \quad ex_1 \rightarrow ex_2) \\
ex_0 \rightarrow (ex_1\ ex_2) \quad&\triangleq&\quad (ex_0 \rightarrow ex_1 \quad ex_0 \rightarrow ex_1)
\end{eqnarray*}

Similarly for $\top$, this refers to every possible expression in the universe.
Thus, putting $\top$ into the same environment as some other expression renders the latter expression redundant.

\begin{eqnarray*}
(\top\ \top) \quad&\triangleq&\quad \top \\
(ex_0\ ex_1\ \top\ ex_3) \quad&\triangleq&\quad \top
\end{eqnarray*}

It is perhaps not immediately obvious at this point (because context has not been discussed yet), but any expression that has $\top$ in its domains is equivalent to $\top$.
This is because $\top$ will be absorbed into the context of its codomain once the expression is evaluated -- this will become more clear taking into account the rules that deal with context.

\begin{eqnarray*}
\top \rightarrow exs \quad&\triangleq&\quad \top \\
((\top \rightarrow exs_0) \rightarrow exs_1) \rightarrow exs_2 \quad&\triangleq&\quad \top \\
(exs_0 \rightarrow \top) \rightarrow exs_1 \quad&\not\triangleq&\quad \top
\end{eqnarray*}

Every sub-expression of a larger expression also has the parent expression's context.
However, context is created when an arrow is traversed, hence sub-expressions occuring after one or more arrows will have additional context (once the arrow has been evaluated).

\begin{eqnarray*}
ctx \vdash exs_0 \rightarrow exs_1 \quad&\triangleq&\quad (ctx \vdash exs_0) \rightarrow exs_1 \\
ctx \vdash (exs_0 \rightarrow exs_1) \rightarrow exs_2 \quad&\triangleq&\quad (ctx \vdash exs_0 \rightarrow exs_1) \rightarrow exs_2 \\
                                                            &\triangleq&\quad ((ctx \vdash exs_0) \rightarrow exs_1) \rightarrow exs_2 \\
ctx \vdash exs_0.exs_1 \quad&\triangleq&\quad (ctx \vdash exs_0).exs_1 \\
ctx \vdash exs_0.exs_1.exs_2 \quad&\triangleq&\quad (ctx \vdash exs_0.exs_1).exs_2 \\
                                  &\triangleq&\quad (ctx \vdash exs_0).exs_1.exs_2
\end{eqnarray*}

Finally atoms cannot exist twice in the same collection, they are unique elements. Thus,

\begin{eqnarray*}
(\mathtt{t}\ \mathtt{t}) \quad&\triangleq&\quad \mathtt{t} \\
(\mathtt{t_a}\ ex_0\ \mathtt{t_b}\ \mathtt{t_b}\ ex_1\ \mathtt{t_a}\ ex_2\ \mathtt{t_c}) \quad&\triangleq&\quad (\mathtt{t_a}\ \mathtt{t_b}\ \mathtt{t_c}\ ex_0\ ex_1\ ex_2) \\
(\mathtt{t} \rightarrow exs_0 \quad \mathtt{t} \rightarrow exs_1) \quad&\triangleq&\quad \mathtt{t} \rightarrow (exs_0\ exs_1) \\
(\mathtt{t} \rightarrow exs_0 \quad \mathtt{t}) \quad&\triangleq&\quad (\mathtt{t} \rightarrow exs_0) \\
(\mathtt{t_a} \rightarrow \mathtt{t_b} \rightarrow exs_0 \quad \mathtt{t_a} \rightarrow \mathtt{t_b}) \quad&\triangleq&\quad (\mathtt{t_a} \rightarrow \mathtt{t_b} \rightarrow exs_0) \\
(\mathtt{t_a} \rightarrow (\mathtt{t_b}\ \mathtt{t_c}) \rightarrow exs_0 \quad \mathtt{t_a} \rightarrow \mathtt{t_b} \rightarrow exs_1) \quad&\triangleq&\quad (\mathtt{t_a} \rightarrow \mathtt{t_b} \rightarrow (exs_0\ exs_1) \quad \mathtt{t_a} \rightarrow \mathtt{t_c} \rightarrow exs_0)\\
    \quad&\triangleq&\quad (\mathtt{t_a} \rightarrow (\mathtt{t_b} \rightarrow (exs_0\ exs_1) \quad \mathtt{t_c} \rightarrow exs_0))
\end{eqnarray*}
Also note the previous rule for $ex_0 \rightarrow (ex_1\ ex_2)$ which is a little more general but relies on the notion of equality for expressions which has some caveats. See, for example, rule 1.4.

\section{Operational Semantics}
The rules for evaluating the operational semantics of \textsl{LangLang} expressions are slightly changed from classical natural deduction.\\

In particular, the way that the sequent $\vdash$ is interpreted is a little different from the usual interpretation although it realizes many of the same objectives in spirit.
The meaning of the sequent in this calculus is the provision of a operational `\emph{context}' in which to evaluate expressions\footnote{Although this sounds similar to the usual convention, the sequent is involved in the evaluation in a far more direct manner.}.
It is also what allows \textsl{LangLang} to perform evaluation in a lazy manner where required\footnote{This also appears to be a (in my humble opinion, fascinating) prerequisite for Turing Completeness in this calculus.}.
The way that the sequent works in the operational semantics is to move inward when evaluating sub-expressions as can be seen in rules 1.4, 2.2, 3.1 etc... but after the evaluation is completed, sub-expressions are moved over into the context (and replaced by a newly generated symbol $\lambda$) as in rules 2.1.1 , 2.1.2, 2.3, 2.4 and (TODO: formerly 2.5?)\footnote{See also the denotational invariants for contexts, documented earlier.}.

\subsection{Ideal Semantics}
\subsubsection{Rule 1. $ctx \vdash exs$ }
\[
\infer{\vdash \bot}{ctx \vdash \bot} \eqno{1.1}
\]

\[
\infer{\vdash \top}{ctx \vdash \top} \eqno{1.2}
\]

\[
\infer{ctx \vdash \lambda}{ctx \vdash \lambda} \eqno{1.3}
\]

\[
\infer{ctx \vdash \mathtt{t}}{ctx \vdash \mathtt{t}} \eqno{1.4}
\]

\[
\infer{ \left(\dfrac{ctx \vdash e}{\vdots}\quad\dfrac{ctx \vdash es}{\vdots}\right) }{ ctx \vdash (e\ es) } \eqno{1.5}
\]
Recall from the introduction that the sequent always moves inward\footnote{Conveniently the sequent is always either on the outside, or only one level deep in the left-hand side of the expression. This fortunate invariant is helpful for simplifying the implementation.}.\\

While this rule is sufficient in the calculus, \textsl{LangLang} extends it to also use the current collection as part of the context.
This allows \textsl{LangLang} to use a similar scoping mechanism to conventional programming languages where declarations inside the same scope are interdependent.
Unfortunately, these dependencies can be quite complex, for example consider the expression  $(a.b \rightarrow c \quad a \rightarrow d.b \quad d \rightarrow b)$.
Therefore this rule may need to be implemented using a more sophisticated mechanism:

\[
\infer{\left(
  \dfrac{ctx \vdash e_0}{\vdots} \dots
  \dfrac{ctx \vdash e_n}{\vdots} \quad
  \dfrac{ctx \vdash (e_0 \rightarrow (e_1 \dots e_n)).\top}{\vdots} \dots
  \dfrac{ctx \vdash (e_n \rightarrow (e_0 \dots e_{n-1})).\top}{\vdots}\right)}
{ctx \vdash (e_0 \dots e_n)} \eqno{1.5.a}
\]

\emph{The type checker also has to handle this extension in a special way.
The expression does not automatically fail to type-check if any of its sub-expressions evaluate to $error$.
It is sufficient to check that at least one of $ctx \vdash (e_k \rightarrow (e_0 \dots {e_{k-1}}\ {e_{k+1}} \dots e_n)).\top$ does not evaluate to $error$ or $\bot$.
Except, if all $ctx \vdash (e_k \rightarrow (e_0 \dots {e_{k-1}}\ {e_{k+1}} \dots e_n)).\top$ evaluates to bottom, then the expression also type-checks.}
This rule could also be written as follows, but it is more difficult to implement type checking in this form.

\[
\infer{\left(
  \dfrac{ctx \vdash e}{\vdots} \quad 
  \dfrac{ctx \vdash es}{\vdots} \quad
  \dfrac{ctx \vdash (e \rightarrow es).\top}{\vdots} \quad
  \dfrac{ctx \vdash (es \rightarrow e).\top}{\vdots} \quad
  \dfrac{ctx \vdash ((es \rightarrow e).\top \rightarrow es).\top}{\vdots}
\right)}
{ctx \vdash (e\ es)} \eqno{1.5.a-alt}
\]

%Also, the semantics implicitly assumes that tokens cannot exist twice in a collection so ????
%\[
%\infer{ctx \vdash (\mathtt{t}\ es)}{ctx \vdash (\mathtt{t}\ \mathtt{t}\ es)} \eqno{1.4.a.a}
%\]

\subsubsection{Rule 2. $ctx \vdash exs_0 \rightarrow exs_1$ }

\[
\infer{\left(ctx \vdash exs_0\right) \rightarrow exs_1}{ctx \vdash exs_0 \rightarrow exs_1} \eqno{2.2}
\]
Recall from the introduction that the sequent always moves inward.

\[
\infer{\vdash \bot}{(\vdash \bot) \rightarrow exs_1} \eqno{2.3}
\]

\[
\infer{\vdash \top}{(\vdash \top) \rightarrow exs_1} \eqno{2.4}
\]

\[
\infer{ctx,\mathtt{t}_0 \rightarrow exs_1 \vdash \lambda \rightarrow exs_1}{(ctx \vdash \mathtt{t}_0) \rightarrow exs_1} \eqno{2.5}
\]

\[
\infer{\left(\dfrac{(ctx_0 \vdash e_{0_0}) \rightarrow exs_1}{\vdots}\ \dots\ \dfrac{(ctx_n \vdash e_{0_n}) \rightarrow exs_1}
{\vdots}\right)}{(ctx_0 \vdash e_{0_0}\ \dots\ ctx_n \vdash e_{0_n}) \rightarrow exs_1} \eqno{2.6}
\]

Note that there will be some implicit special case handling in an implementation for collections that contain $\bot$.
This is not explicitly handled by the operational semantics as we're not concerned with the physical storage of collections.
Using the denotational semantics note the following two implicit rules employed by 2.6.
\[
\infer{ctx_0 \vdash ex_0 \quad ctx_1 \vdash ex_1 \quad ctx_3 \vdash ex_3}
{ctx_0 \vdash ex_0 \quad ctx_1 \vdash ex_1 \quad \vdash \bot \quad ctx_3 \vdash ex_3} \eqno{2.6.a}
\]
\[
\infer{\vdash \bot}{\vdash \bot \quad \vdash \bot} \eqno{2.6.b}
\]

\subsubsection{Rule 3. $ctx \vdash exs_0.exs_1$ }

\[
\infer{\dfrac{(ctx \vdash exs_0).exs_1}{\vdots}}
{ctx \vdash exs_0.exs_1} \eqno{3.1.1}
\]

\[
\infer{\dfrac{\left(\dfrac{ctx \vdash exs_{0_0}.exs_{0_1}}{\vdots}\right).exs_1}{\vdots}}
{(ctx \vdash exs_{0_0}.exs_{0_1}).exs_1} \eqno{3.1.2}
\]

\[
\infer{\left(\lambda \rightarrow \left(\dfrac{ctx \vdash exs_0}{\vdots}\right)\right).\left(\dfrac{ctx \vdash exs_1}{\vdots}\right)}
{(ctx \vdash \lambda \rightarrow exs_0).exs_1} \eqno{3.1.3}
\]

\[
\infer{\vdash \bot}
{(\vdash \mathtt{t}_0).(ctx_1 \vdash exs_1)} \eqno{3.1.4.1 (formerly 3.5.1)}
\]

\[
\infer{\dfrac{\left(\dfrac{(\lambda \rightarrow (\vdash c)).(\mathtt{t}_0 \rightarrow \top)}{\vdots}\right).(ctx_1 \vdash exs_1)}{\vdots}}
{(c \vdash \mathtt{t}_0).(ctx_1 \vdash exs_1)} \eqno{3.1.4.2 (formerly 3.5.2)}
\]

\[
\infer{\dfrac{\left(\dfrac{(\lambda \rightarrow (cs \vdash c)).(\mathtt{t}_0 \rightarrow \top)}{\vdots}\right).(ctx_1 \vdash exs_1)}{\vdots}\quad\dfrac{(cs \vdash \mathtt{t}_0).(ctx_1 \vdash exs_1)}{\vdots}}
{(cs,c \vdash \mathtt{t}_0).(ctx_1 \vdash exs_1)} \eqno{3.1.4.3 (formerly 3.5.3)}
\]


\[
\infer{ctx_0 \vdash \mathtt{t_a}}{(\lambda \rightarrow (ctx_0 \vdash \mathtt{t_a})).(ctx_1 \vdash \mathtt{t_a})} \eqno{3.2}
\]



\subsubsection{OLD Rule 3. $ctx \vdash exs_0.exs_1$ }

%\[
%\infer{\dfrac{(ctx_0 \vdash exs_0).exs_1}{\vdots}}{(ctx_0 \vdash exs_0).\dfrac{\dfrac{(ctx_1 \vdash exs_1)}{\vdots}}{exs_{1_{result}}}} \eqno{3.a}
%\]
%Once the right-hand side of a query has been fully evaluated, its context is discarded.

\[
\infer{\dfrac{\left(\dfrac{ctx \vdash exs_0}{\vdots}\right).\left(\dfrac{ctx \vdash exs_1}{\vdots}\right)}{\vdots}}{ctx \vdash exs_0.exs_1} \eqno{3.1}
\]
Recall from the introduction that the sequent always moves inward (and notice the similarity with rule 2.2).
The right-hand side must also be evaluated before the query can be executed. 
This is the case even when the left-hand side is $\top$ or $\bot$ since type-checking must still be performed on the right-hand side even in these special cases.

\[
\infer{\vdash \bot}{(\vdash \bot).(ctx_1 \vdash exs_1)} \eqno{3.2}
\]
Type checking on the right-hand side for this rule (and later rules) will already be taken care of in rule 3.1. 
Also, evaluated context on the right-hand side is discarded.
\emph{TODO: Note that right-hand side context may also be nested inside $exs_1$... which is a missing detail in this rule.}

\[
\infer{\top \vdash exs_1}{(ctx_0 \vdash \top).(ctx_1 \vdash exs_1)} \eqno{3.3}
\]
Also, evaluated context on the right-hand side is discarded.
\emph{TODO: Note that right-hand side context $ctx_1$ may also be nested inside $exs_1$... which is a missing detail in this rule.}


\[
\infer{\vdash \bot}{(ctx_0 \vdash \lambda).(ctx_1 \vdash exs_1)} \eqno{3.4}
\]

\[
\infer{\vdash \bot}{(\vdash \mathtt{t}_0).(ctx_1 \vdash exs_1)} \eqno{3.5.1}
\]

\[
\infer{\dfrac{\left(\dfrac{(\lambda \rightarrow (\vdash c)).(\mathtt{t}_0 \rightarrow \top)}{\vdots}\right).(ctx_1 \vdash exs_1)}{\vdots}}{(c \vdash \mathtt{t}_0).(ctx_1 \vdash exs_1)} \eqno{3.5.2}
\]

\[
\infer{\dfrac{\left(\dfrac{(\lambda \rightarrow (cs \vdash c)).(\mathtt{t}_0 \rightarrow \top)}{\vdots}\right).(ctx_1 \vdash exs_1)}{\vdots}\quad\dfrac{(cs \vdash \mathtt{t}_0).(ctx_1 \vdash exs_1)}{\vdots}}
{(cs,c \vdash \mathtt{t}_0).(ctx_1 \vdash exs_1)} \eqno{3.5.3}
\]

This rule operates on the assumption that all expression in the context has already been evaluated.

\[
\infer{\dfrac{\left(\dfrac{ctx \vdash exs_{0_0}.exs_{0_1}}{\vdots}\right).exs_1}{\vdots}}
{ctx \vdash (exs_{0_0}.exs_{0_1}).exs_1} \eqno{3.6 (TODO)}
\]

\[
\infer{\dfrac{ctx \vdash exs_0.\left(\dfrac{ctx \vdash exs_{1_0}.exs_{1_1}}{\vdots}\right)}{\vdots}}
{ctx \vdash exs_0.(exs_{1_0}.exs_{1_1})} \eqno{3.6 (TODO)}
\]

\[
\infer{(exs_{0_0}.exs_{1_0} \rightarrow (exs_{0_0} \rightarrow exs_{0_1}).\top.((exs_{1_0} \rightarrow exs_{1_1}).\top)).\top}
{(\lambda \rightarrow exs_{0_0} \rightarrow exs_{0_1}).(exs_{1_0} \rightarrow exs_{1_1})} \eqno{3.6 (TODO)}
\]

\[
\infer{}
{(ctx \vdash exs_{0_0} \rightarrow exs_{0_1}).(exs_{1_0} \rightarrow exs_{1_1})} \eqno{3.6 (TODO)}
\]

\[
\infer{ctx \vdash \lambda \rightarrow exs_1}
{ctx \vdash (\lambda \rightarrow exs_0) \rightarrow exs_1} \eqno{2.1.1(TODO)}
\]

Note that the $\lambda$ is never captured inside the context as it is a pseudo-token\footnote{Conceptually $\lambda$ behaves like any regular symbol, thus a new unique symbol could be generated and placed in the context.
Following this conceptual framework, the evaluator would have to have the ability to assign a new unique symbols for each new instance of $\lambda$ it generates automatically -- almost a little bit like a turing machine writing to its infinite length tape except that no cell can be destructively updated.}.

\[
\infer{ctx, \mathtt{t}_0 \rightarrow exs_0 \vdash \lambda \rightarrow exs_1}
{ctx \vdash (\lambda \mathtt{t}_0 \rightarrow exs_0) \rightarrow exs_1}  \eqno{2.1.2(TODO)}
\]

This form of the lambda expression is just syntactic sugar, but it is very helpful in some cases.

\[
\infer{\dfrac{\left(\lambda \rightarrow \left(\dfrac{ctx_0 \vdash exs_{0_1}}{\vdots}\right)\right).(ctx_1 \vdash exs_1)}{\vdots}}{(ctx_0 \vdash \lambda \rightarrow exs_{0_1}).(ctx_1 \vdash exs_1)} \eqno{3.6.1}
\]

\[
\infer{ctx \vdash exs_{0_1}}{(\lambda \rightarrow (ctx \vdash exs_{0_1})).(\vdash \top)} \eqno{3.6.2}
\]
Not that the context could also be nested inside $exs_{0_1}$, this isn't explicitly shown...

\[
\infer{\left(\dfrac{(\lambda \rightarrow (ctx_{0_0} \vdash e_{0_1})).(ctx_1 \vdash exs_1)}{\vdots} \dots \dfrac{(\lambda \rightarrow (ctx_{0_n} \vdash e_{0_n})).(ctx_1 \vdash exs_1)}{\vdots}\right)}
{(\lambda \rightarrow (ctx_{0_0} \vdash e_{0_1} \dots ctx_{0_n} \vdash es_{0_n})).(ctx_1 \vdash exs_1)} \eqno{3.6.3}
\]
Technically right-hand side context can be discarded entirely, just keeping the sequent itself for matching with the other rules that expext to take a query with internal context on the right-hand side.

\[
\infer{\left(\dfrac{(\lambda \rightarrow (ctx_0 \vdash ex_0)).(ctx_{1_0}\vdash e_{1_0})}{\vdots} \dots \dfrac{(\lambda \rightarrow (ctx_0 \vdash ex_0)).(ctx_{1_n} \vdash es_{1_n})}{\vdots}\right)}
{(ctx_0 \vdash \lambda \rightarrow ex_0).(ctx_{1_0} \vdash e_{1_0} \dots ctx_{1_n} \vdash e_{1_n})} \eqno{3.6.4}
\]
Keeping the sequent syntax but discarding the internal context, same as rule 3.6.2.

\emph{TODO: it's likely the internal context of the right-hand side might need to be retained in order to test equality of functions... (through co-evalation?).
In this case we'll likely need to re-evaluate context being discarded in previous rules...}


\[
\infer{ctx_0 \vdash \mathtt{t_a}}{(\lambda \rightarrow (ctx_0 \vdash \mathtt{t_a})).(ctx_1 \vdash \mathtt{t_a})} \eqno{3.7.1}
\]

\[
\infer{ctx_0 \vdash \mathtt{t_a} \rightarrow exs_0}{(\lambda \rightarrow (ctx_0 \vdash \mathtt{t_a} \rightarrow exs_0)).(ctx_1 \vdash \mathtt{t_a})} \eqno{3.7.2}
\]

\[
\infer{\mathtt{t_a} \rightarrow \left(\dfrac{(\lambda \rightarrow (ctx_0,\mathtt{t_a} \rightarrow exs_0 \vdash exs_0)).\left(\dfrac{ctx_1,\mathtt{t_a} \rightarrow exs_1 \vdash exs_1}{\vdots}\right)}{\vdots}\right)}
{(\lambda \rightarrow ((ctx_0 \vdash \mathtt{t_a}) \rightarrow exs_0)).((ctx_1 \vdash \mathtt{t_a}) \rightarrow exs_1)} \eqno{3.7.3}
\]
And the context on the right-hand side of the arrow can be discarded... \emph{TODO: But rule 2.5 might be a problem on the right-hand side of the query!!! can't replace $t_a$ with a lambda here!
It appears as if 2.5 should only be evaluated on the left-hand side of a query, however this is easier said than done.}

\[
\infer{todo...??}
{(\lambda \rightarrow (((ctx_0 \vdash \mathtt{t_a}) \rightarrow exs_{0_0}) \rightarrow exs_{0_1})).((ctx_1 \vdash \mathtt{t_a}) \rightarrow exs_1)} \eqno{3.7.4}
\]

\[
\infer{todo...??}
{(\lambda \rightarrow (((ctx_0 \vdash \mathtt{t_a}) \rightarrow exs_{0_0}) \rightarrow exs_{0_1})).((ctx_1 \vdash \mathtt{t_a}) \rightarrow exs_1)} \eqno{3.7.4}
\]


\[
\infer{ctx, \mathtt{t}_{0_0} \rightarrow exs_{0_1}\vdash exs_{0_1}}{ctx \vdash (\mathtt{t}_{0_0} \rightarrow exs_{0_1}).\top} \eqno{3.1.4.1.1}
\]

\[
\infer{ctx \vdash \left(\left(\lambda \rightarrow \dfrac{ctx \vdash exs_{0_{0_0}} \rightarrow exs_{0_{0_1}}}{\vdots}\right).\top \rightarrow exs_{0_1}\right).\top}
{ctx \vdash ((exs_{0_{0_0}} \rightarrow exs_{0_{0_1}}) \rightarrow exs_{0_1}).\top} \eqno{3.1.4.1.3}
\]
\emph{TODO: This is still not correct... need some way to evaluate the first (nested) arrow only once, without going into an infinite loop.}

\[
\infer{\left(\dfrac{ctx,(e_{0_0}\ e_{0_1}\ \dots\ e_{0_n}) \vdash e_{0_0}.exs_1}{\vdots}\ \dots\ \dfrac{ctx,(e_{0_0}\ e_{0_1}\ \dots\ e_{0_n}) \vdash e_{0_n}.exs_1}{\vdots}\right)}
{ctx \vdash (e_{0_0}\ e_{0_1}\ \dots\ e_{0_n}).exs_1} \eqno{3.1.5}
\]

\emph{TODO: Check rule 3.1.5, there was still some confusion here about evaluation and possible memoization - see the code (it's 3.1.4 in the code at the moment)}

\subsection{Implementation Semantics}

\section{Implementation Details}

\section*{Acknowledgements}

\bibliographystyle{plain}
\bibliography{specification-0.1}

\end{document}
