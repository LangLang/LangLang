\documentclass[a4paper,11pt]{article}

%%%%%%%%%%%%%%%%%%% PACKAGES %%%%%%%%%%%%%%%%%%%
%\usepackage{tikz}
%\usepackage{geometry}
%\geometry{a4paper}
%\usepackage{gcl}
%\usepackage{float}
%\usepackage{wrapfig}
%\usepackage{proof}
\usepackage{amsmath}
\usepackage{amssymb}
%\usepackage{amsthm}

%%%%%%%%%%%%%%%%%%%% MACROS %%%%%%%%%%%%%%%%%%%%

%%%%%%%%%%%%%%%%%%% DOCUMENT %%%%%%%%%%%%%%%%%%%
\begin{document}

\title{LangLang 0.1}
%\subtitle{Specification}
\author{Rehno Lindeque}

\maketitle

\begin{abstract}
A formal specification of the syntax and semantics for the programming language \textsl{LangLang}.
\end{abstract}
\textbf{Keywords:} \textsl{LangLang}, specification, denotational semantics, operational semantics

\section{Introduction}

\section{Syntax Specification}
\subsection{Formal syntax}

\subsection{Conventions used in this document}
\label{sec:conventions}

Normally \textsl{LangLang}'s syntax dictate that \emph{bottom} should be written as an empty pair of parentheses \texttt{()} to indicate that \emph{bottom} is formally just an empty collection in \textsl{LangLang}.
In this document \emph{bottom} will usually be written using the standard mathematical notation $\bot$.
However, occasionaly the standard syntax for \emph{bottom}, $()$ may be used to emphasize an empty collection.
Similarly \emph{top}, written using an underscore \texttt{\_} in LangLang, corresponds to the mathematical notation $\top$ used in this document.\\

In addition the following conventions are used to indicate the nature of syntactic elements in the semantics.

\begin{description}
  \item[$exs$] A collection of expressions (which may also be, depending on circumstance, $\top$, $\bot$ or a single expression).
  \item[$ex$] A collection consisting of a single expression.
  \item[\texttt{tok}] A token representing some atom.
\end{description}

\section{Denotational Semantics}

The denotational semantics listed here serve as invariants that inform the design of the operational semantics in the next section.\\

Atoms are the most basic elements in \textsl{LangLang} and are represented by simple whitespace free \emph{tokens} in source code.
In denotational terms two atoms are equal if their token representation is identical.
To facilitate generic programming tokens may also be constructed programmatically, however we delay this topic for later sections.\\

Collections in \textsl{LangLang} do not have any \emph{structure} as all structure is derived using the concept of arrows.
Intuitively this means that collections are fully defined by their contents and have transient ``boundaries'', so to speak.
They cannot be nested in the sense that any collection that contains another collection is indistinguishable from the union of all their elements.
This is unlike the common notion of \emph{sets} in mathematics which includes the set as a basic mathematical object with its own structure.\\

Syntactically, parentheses are used to group things together in \textsl{LangLang}, thus collections may be represented in code by enclosing their contents.
However, as mentioned above, collections' boundaries are transient and disappear when nested.
\begin{eqnarray}
(ex0\ (ex1\ (ex2\ ex3))\ ex4) \quad\triangleq\quad (ex0\ ex1\ ex2\ ex3\ ex4)
\end  {eqnarray}

Furthermore, when a collection contains only a single expression, parentheses may be dropped entirely.
\begin{eqnarray}
(ex) \quad\triangleq\quad ex
\end  {eqnarray}

An empty collection $()$ is denoted, in this document, with the symbol $\bot$\footnote{Please take note of the conventions for writing \emph{bottom} mentioned in section \ref{sec:conventions}.}.
\begin{eqnarray}
() \quad\triangleq\quad \bot
\end{eqnarray}

An arrow is an \emph{operational} construct which adds structure to the only other objects in the language, atoms.
As collections are not objects, arrows may not connect to them directly, but instead connect only to their contents.
For this reason, an arrow that connects to an empty collection does not exist.
\begin{eqnarray}
\bot \rightarrow exs \quad\triangleq\quad \bot \\
exs \rightarrow \bot \quad\triangleq\quad exs \\
(ex0\ ex1) \rightarrow ex2 \quad\triangleq\quad (ex0 \rightarrow ex2 \quad ex1 \rightarrow ex2) \\
ex0 \rightarrow (ex1\ ex2) \quad\triangleq\quad (ex0 \rightarrow ex1 \quad ex0 \rightarrow ex1)
\end{eqnarray}

\section{Operational Semantics}
\subsection{Ideal Semantics}

\subsection{Implementation Semantics}

\section{Implementation Details}

\section*{Acknowledgements}

\bibliographystyle{plain}
\bibliography{specification-0.1}

\end{document}
