
\documentclass[11pt]{article}

%%%%%%%%%%%%%%%%%%% PACKAGES %%%%%%%%%%%%%%%%%%%
\usepackage{geometry}
\geometry{a4paper}
%\usepackage{gcl}
%\usepackage{pfs}
%\usepackage{algorithmic}
%\usepackage{graphicx}
%\usepackage{epstopdf}
%\usepackage{natbib}
%\usepackage{fancyvrb}
\usepackage{float}
%\usepackage{wrapfig}
%\usepackage{proof}
\usepackage{verbatim} 

\usepackage{amssymb}
\usepackage{amsthm}
\usepackage{amsmath}
\usepackage{stmaryrd}

%\usepackage{listings}
\usepackage{framed}
%\usepackage{caption}
%\usepackage{algo}
%\usepackage{wsuipa}
%\usepackage{ipalmacs}
%\usepackage{textcomp}
\usepackage{tocloft}
\usepackage{listing}


%%%%%%%%%%%%%%%%%%%% MACROS %%%%%%%%%%%%%%%%%%%%

%\floatstyle{boxed}
%\newfloat{example}{thp}{lxp}
%\floatname{example}{Example}

\newcommand{\listofexamplestitle}{List of Examples}
\newlistof{example}{exp}{\listofexamplestitle}

\newcommand{\example}[1]{%
      \refstepcounter{example}
      \par\noindent\caption{#1}
      \addcontentsline{exp}{example}
      {\protect\numberline{\theexample}#1}\par}

\floatstyle{boxed}
\newfloat{program}{thp}{lop}
\floatname{program}{Example}

%%%%%%%%%%%%%%%%%%%%%Stuff from SICL article%%%%%%

%%%%%%%%%%%%%%%%%%% DOCUMENT %%%%%%%%%%%%%%%%%%%
\begin{document}

\title{LangLang Specification\\\small{version 0.1}}
\author{Rehno Lindeque}

\maketitle

\tableofcontents

\section{Introduction}
This document is intended as a specification for implementers to write implementations against. For language design and rationalle see \cite{ctxfoundation} and \cite{ctxrationalle}.

\section{Overview}

\section{General structure}
\subsection{Source code structure}

\subsection{Build system structure}

\subsection{Syntactic Issues}

\section{Semantics}
\subsection{Definitions}

\subsection{Recursive definitions}

\subsection{Selection queries}

\subsection{Mutation queries}

\subsection{Stateful queries}

\section{Pragmatics}
\subsection{Processes, External calls, Non-determinism}

\subsection{Distributed processes}

\subsection{Live updates}

\subsection{Proofs}

\section{Issues}

\paragraph{Issue 1.1 -- Should nested contexts overide parent contexts or only add information?}

\subparagraph{Status:} Open

\subparagraph{Discussion (2010-02-06):}
Suppose 2 nested contexts, $a$ and $b$ are defined with a simple relation $x$ in each as illustrated bellow.
\begin{program}[H]
$
  a \rightarrow \left [ \begin{array}{l}
    x \rightarrow 4 \\
    b \rightarrow [x \rightarrow 5]
  \end{array} \right ] \\\\
  a:b:x:5 \text{\quad\{ This will always be true. \}} \\
  a:b:x:4 \text{\quad\{ But does x inherrit the relation in $a:x$ as well? \}}
$
\example{Nested domains with similar variables}
\end{program}

A user may expect nesting to override values defined in a parent context, however it seems desirable to have the ability to extend sets from parent contexts.

Futhermore, consider cases in which we expect a relation to be inherited from a parent context.
\begin{program}[H]
$
  a \rightarrow \left [ \begin{array}{l}
    x \rightarrow 4 \\
    b \rightarrow [x]
  \end{array} \right ] \\\\
  a:b:x:4 \text{\quad\{ True? \}}
$
\example{Automatic implication inheritance}
\end{program}

This becomes more complicated in a case such as this.
\begin{program}[H]
$
  a \rightarrow \left [ \begin{array}{l}
    x \rightarrow 4 \\
    b \rightarrow 
    \left [ \begin{array}{l}
      x \rightarrow 5 \\
      c \rightarrow x
    \end{array} \right ]
  \end{array} \right ] \\\\
  a:b:c:x:5 \text{\quad\{ True? \}} \\
  a:b:c:x:4 \text{\quad\{ True? \}}
$
\example{Automatic implication inheritance}
\end{program}

\subparagraph{Resolved (2010-02-06 to 2010-02-07):}
Let us go back to the logical background of contexts.
If we say that the symbol $\rightarrow$ means `implies' then $x \rightarrow 4$ and $b \implies x$ in the same context will tell us denotationaly that $b \rightarrow 4$. 
(Furthermore the type system will tell us that $b$ implies that $4$ is implied by $x$. I.e. $b \rightarrow $ $_{x \rightarrow}4$.)
Now, suppose one additional rule $c \rightarrow [ x \rightarrow 5 ]$ is added to the same context.
Then $c$ implies that $x$ implies $5$. Does $x$ imply both $4$ and $5$ in the context of $c$?


Consider a query definition $i \rightarrow b.x$. Translated into english $i$ implies whatever is implied by the predicate $b.x$ (I.e. ``x exists in b''). 
Because if $x$ did not exist in $b$ then we would not expect $i$ to hold the value $4$. 
The fact that we select $x$ from $b$ seems to imply that we are only looking for the specific $x$ that satisfies the type $_{b \rightarrow }x$. 
Once the selection returns $_{b \rightarrow }x$ we can look up the symbol's codomain lazily if any exists. 
In order to select all $x$ we could construct $i$ as $i \rightarrow [ x\:\: b.x ]$, hence this method is still general.


However, suppose that we selected $x$ from yet another definition in the contex: $d \rightarrow x$. 
Now the query $j \rightarrow d.x$ gives us the typed result $_{d \rightarrow } x$. 
However this symbol has no codomains.
Our definition of an arrow (the symbol $\rightarrow$) is based on boxing and unboxing. 
An arrow defines a \em{labeled box} which is automatically unboxed by the compiler in a lazy manner.
Hence if no arrow exists, then the compiler will look up one in a parent context.
Thus, $_{d \rightarrow } x$ inherits the relations of $x$ in the parent context and $j:d:x:4$ holds.
To create a boxed `empty set' that does not inherit from a parent context we simply use the notation $x \rightarrow []$.

\subparagraph{Discussion (2010-02-07):} 
How can we extend a symbol in the parent context with additional relations in the child context?

(See ``Unboxing is Evaluation'')

\subparagraph{Resolved (2010-02-07):}
Perhaps it is possible to use the same `preceding domain' syntax for parent selection as we use for recursion. 
Take the example below.
\begin{program}[H]
$
  x \rightarrow 4\\
  a \rightarrow  [ x \rightarrow  [ 5 \:\: @:x ]]\\
  a:x:5 \text{\quad\{ True \}} \\
  a:x:4 \text{\quad\{ True \}}
$
\example{Explicit inheritance}
\end{program}

\paragraph{Issue 1.2 -- Perhaps define operators $..$ and $::$ as `deep' selection?}

\subparagraph{Status:} Open
\subparagraph{References:} Issue 1.1
\subparagraph{Discussion (2010-??-??):}

\newpage
\listofexample

\listoftables

\listoffigures

\bibliographystyle{plain}
\bibliography{langlangspec01}

\end{document}

